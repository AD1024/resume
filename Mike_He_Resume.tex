\documentclass{resume}
\usepackage{hyperref}

\usepackage[left=0.60in,top=0.3in,right=0.60in,bottom=0.3in]{geometry}
\usepackage{titlesec}
\usepackage{enumitem}
\titlespacing*{\rSection}{0pt}{1.1\baselineskip}{\baselineskip}
% \name{Mike He}
% \address{1135 NE CAMPUS PKWY \\ Seattle, WA 98105}
% \address{(206)~$\cdot$~887~$\cdot$~8588 \\ dh63@cs.washington.edu}
% \address{https://ad1024.space}

\begin{document}
	\MakeUppercase{\Large{\textbf{Mike He}}} \hfill {\em{\href{mailto:dh63@cs.washington.edu}{dh63@cs.washington.edu}}}\\
	\vspace{-5pt}\href{https://ad1024.space}{https://ad1024.space} \hfill{\em (206)~$\cdot$~887~$\cdot$~8588}

%----------------------------------------------------------------------------------------
%	EDUCATION
%----------------------------------------------------------------------------------------

	\begin{rSection}{Education}
	{\bf University of Washington, Seattle} \hfill {\em Sept. 2018---Est. Jun. 2022} \\
	\textit{B.S. in Computer Science}
	\vspace{-5pt}
        \begin{itemize}[leftmargin=*]
            \setlength{\itemsep}{1pt}
            \setlength{\parskip}{0pt}
			\setlength{\parsep}{0pt}
			\item Cumulative GPA: 3.87
            \item Interests of Studies: Programming Languages \& Formal Verification
		\end{itemize}
		\vspace{-5pt}
		\textbf{Selected Courses:}
		\vspace{-5pt}
		\begin{itemize}
			\setlength{\itemsep}{1pt}
            \setlength{\parskip}{0pt}
			\setlength{\parsep}{0pt}
			\item \textbf{CSE 332} Data Structures \& Parallelism
            \item \textbf{CSE 402} Domain-Specific Langauges
            \item \textbf{CSE 490} Programming Languages \& Program Verification
            \item \textbf{CSE 507} Computer-Aided Reasoning for Software
            \item \textbf{CSE 505} Principle of Programming Languages
		\end{itemize}
	\end{rSection}
	\vspace{-5pt}

%----------------------------------------------------------------------------------------
%	Specifications
%----------------------------------------------------------------------------------------

	\begin{rSection}{Skills}
		\begin{tabular}{ @{} >{\bfseries}l @{\hspace{4ex}} l }
			Languages & Python, Java, Haskell, Agda, OCaml, \LaTeX \\
			Skills    & Certified Programming, Functional Programming, Automated Verification \\
			What's more?    & I've been playing the violin for 16 years. \\ &I like Symphonies composed by \href{https://imslp.org/wiki/Category:Mahler,_Gustav}{Gustav Mahler}.
		\end{tabular}
	\end{rSection}

%----------------------------------------------------------------------------------------
%	WORK EXPERIENCE
%----------------------------------------------------------------------------------------

	\begin{rSection}{Experience}

		% \begin{rSubsection}{Troph}{Dec. 2019---Now}{Backend Developer Intern}{Remote}
		% 	\item Being responsible for developing server Backend and populate REST API for the Troph IRC website.
		% \end{rSubsection}

	%------------------------------------------------ 

		\begin{rSubsection}{PLSE}{Oct. 2019---Now}{Research Assistant}{Seattle, WA} 
			\item Working on Project Marlowe, synthesizing compilers for domain-specific hardwares. (Summer Research Intern).
			\item Worked on tvm/relay, a High-Level Intermidiate Representation for \href{https://tvm.ai/}{tvm}.
			\item Worked on Dynamic Tensor Rematerialization, an online greedy gradient checkpointing algorithm.
		\end{rSubsection}
		\vspace{-5pt}

	%-------------------------------------------------

		\begin{rSubsection}{UWECE}{Jan. 2019---Sept. 2019}{Research Assistant}{Seattle, WA}
			\item Developed an online panel for visualizing data collected from solar panels deployed around UW campus; Deployed the system to the CEI Testbeds, tested for its stability and optimized its performance.
		\end{rSubsection}
		\vspace{-5pt}

	%------------------------------------------------

		% \begin{rSubsection}{HCC Computing Community}{Feb. 2015--Mar. 2018}{President \& Lecturer \& Web Developer}{Bejing, China}
		% 	\item Developed \& maintained a website named Shiyiquan for Beijing National Day School to manage student organizations. Designed courses and lectures for high school students to learn Python, Android development and functional programming and held classes for students interested in computer programming.
		% \end{rSubsection}
		% \vspace{-5pt}
	\end{rSection}
	\vspace{-5pt}
	% \newpage
	%----------------------------------------------------------------------------------------
	%	PROJECTS
	%----------------------------------------------------------------------------------------

	\begin{rSection}{Projects}
	
        %------------------------------
        %           BNDSOJ
        %------------------------------
		% \textbf{BNDS Online Judge} \hfill {\em {\href{https://github.com/AD1024/BNDSOJ}{On Github}}}
		% \vspace{-5pt}

		% An online judge designed for Beijing National Day School, developed based on \href{https://github.com/vfleaking/uoj}{Universal Online Judge (UOJ)}. It is optimized for CS course instructors to assign, review and grade homework for students. Implementations are contributed to UOJ open source community. Currently, it is being used as the official testing website at BNDS.
		% \vspace{-5pt}

		%------------------------------
        %          Realm.js
        %------------------------------
		% \textbf{Realm.js}
		% \vspace{-5pt}

		% A functional reactive programming library designed for frontend developers. Written in TypeScript, the library combines the design of Elm and RxJs. It adopts a functional imperative and reactive hybrid paradigm of frontend programming. It is the final project of CSE 402 (Domain-Specific Language).
		% \vspace{-5pt}

		%------------------------------
		%            Sager
		%------------------------------
		\textbf{Sager} \hfill {\em {\href{https://github.com/AD1024/Sager}{On Github}}}
		\vspace{-5pt}

		A Demonic Graph Synthesizer (data structure synthesizer) that aims for worst-case performance of graph-related algorithms (e.g. SSSP). It uses \href{https://github.com/emina/rosette}{\textsc{Rosette}} as the solver for synthesizing the core structure and calls a scaler for generating larger graphs. It is the final project of CSE 507 (Computer-Aided Reasoning for Software).
		% \vspace{-5pt}

		%------------------------------
		%      torch checkpointing
		%------------------------------
		\textbf{veripy} \hfill {\em {\href{https://github.com/AD1024/veripy}{On Github}}}
		\vspace{-5pt}

		An auto-active verification library for Python3 base on verification condition generation of Hoare Logic.

		%------------------------------
		%            Blog
		%------------------------------
		% \textbf{Blog} \hfill {\em {\href{https://github.com/BNDS-Programmers/Blog}{On Github}}}
		% \vspace{-5pt}

		% A general-purpose blog system built by using Koa2 that supports both RTF and Markdown; renders mathematical expressions using KaTeX; supports code highlight with hightlight.js; has multi-user management functionality; provides customization supports for about page. 

        %------------------------------
        %         Lotus Leaf
        %------------------------------
		% \textbf{Lotus Leaf} \hfill {\em {\href{https://github.com/AD1024/lotus-leaf-frontend}{On Github}}}
		% \vspace{-5pt}

		% A frontend of a data panel for visualizing and monitoring data sets collected from solar metrics deployed arround University of Washington campus; The system is deployed at the Clear Energy Institute at University Village. This is a research project of ECE Department instructed by Professor Kirschen.

	\end{rSection}
	\vspace{-5pt}
	
	%----------------------------------------------------------------------------------------
	%	HONORS
	%----------------------------------------------------------------------------------------

	\begin{rSection}{Honors}
		\begin{itemize}
			\setlength{\itemsep}{1pt}
			\setlength{\parskip}{0pt}
			\setlength{\parsep}{0pt}
			\item \textbf{Lynn Conway Research Award (DTR Team)}, \href{https://adacenter.org}{ADA} \hfill {\em 2020}
			\item \textbf{JASSO Scholarship}, Waseda University \hfill {\em Jun. 2019}
			\item \textbf{Annual Dean's List}, University of Washington \hfill {\em 2018---2020}
			% \item \textbf{Global Rank 16 (out of 300+)}, High School CTF 5 \hfill {\em May. 2018}
			% \item \textbf{Second Prize}, National Software Development Contest for High School Students \hfill {\em Jul. 2017}
			\item \textbf{First Prize}, Software Engineering Hackathon for High School Students (Beijing) \hfill {\em Apr. 2017}
			% \item \textbf{Global Rank 33 (out of 400+)}, High School CTF 4 \hfill {\em May. 2017}
			\item \textbf{Second Prize}, National Olympiad in Informatics (Beijing Regional) \hfill {\em Dec. 2016}
			% \item \textbf{Outstanding Developer Team}, Xiaomi Inc. \hfill {\em Mar. 2015}
		\end{itemize}
	\end{rSection}
\end{document}
