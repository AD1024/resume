\documentclass{resume}
\usepackage{hyperref}

\usepackage[left=0.40in,top=-0.2in,right=0.40in,bottom=0.3in]{geometry}
\usepackage{titlesec}
\usepackage{enumitem}
\usepackage{natbib, bibentry}
\titlespacing*{\rSection}{0pt}{1.1\baselineskip}{\baselineskip}
% \name{Mike He}
% \address{1135 NE CAMPUS PKWY \\ Seattle, WA 98105}
% \address{(206)~$\cdot$~887~$\cdot$~8588 \\ dh63@cs.washington.edu}
% \address{https://ad1024.space}
\newcommand{\myul}[2][blue]{\setulcolor{#1}\ul{#2}\setulcolor{blue}}
\usepackage{xcolor,soul,lipsum}

\begin{document}
\bibliographystyle{unsrt}
\nobibliography{publication.bib}
	\MakeUppercase{\Large{\textbf{Mike He}}} \hfill {\em{\href{mailto:dh63@cs.washington.edu}{dh63@cs.washington.edu}}}\\
	\vspace{-5pt}\href{https://ad1024.space}{https://ad1024.space} \hfill{\em (206)~$\cdot$~887~$\cdot$~8588}


%----------------------------------------------------------------------------------------
%	Specifications
%----------------------------------------------------------------------------------------

	% \begin{rSection}{Skills}
	% 	\begin{tabular}{ @{} >{\bfseries}l @{\hspace{4ex}} l }
	% 		Languages & C/C++, Python, Java, Rust, OCaml, Haskell, Coq, Agda, \LaTeX \\
	% 		Skills    & Certified \& Functional Programming, Automated Verification \\
	% 		Others    & I've been playing the violin for 17 years. I like Symphonies composed by \href{https://imslp.org/wiki/Category:Mahler,_Gustav}{Gustav Mahler}.
	% 	\end{tabular}
	% \end{rSection}
	% \vspace{-5pt}
%----------------------------------------------------------------------------------------
%	EDUCATION
%----------------------------------------------------------------------------------------

	\begin{rSection}{Education}
	{\bf University of Washington, Seattle} \hfill {\em Sept. 2018---Est. Jun. 2022} \\
	\textit{B.S. in Computer Science}
	\vspace{-5pt}
        \begin{itemize}[leftmargin=*]
            \setlength{\itemsep}{1pt}
            \setlength{\parskip}{0pt}
			\setlength{\parsep}{0pt}
			\item Cumulative GPA: 3.88 / 4
            \item Field of Studies: Programming Languages \& Formal Verification \& Compilers \& MLSys
            \item Selected Coursework: Data Structures \& Parallelism; Intro to Algorithms; Advanced Programming Languages \& Verification; Intro to Artificial Intelligence; System Programming; Computational Complexity
			\item PLs \& Skills: \textbf{Java}, \textbf{Python}, \textbf{C++}, \textbf{Coq}, \textbf{OCaml}, \textbf{Rust}; Algorithm Design, Formal Verification
		\end{itemize}
		% \vspace{-5pt}
		% \textbf{Selected Courses:}
		% \vspace{-5pt}
		% \begin{itemize}
		% 	\setlength{\itemsep}{1pt}
        %     \setlength{\parskip}{0pt}
		% 	\setlength{\parsep}{0pt}
		% 	\item \textbf{CSE 332} Data Structures \& Parallelism
        %     \item \textbf{CSE 505} Principle of Programming Languages
        %     \item \textbf{CSE 507} Computer-Aided Reasoning for Software
		% \end{itemize}
	\end{rSection}
	\vspace{-5pt}

%----------------------------------------------------------------------------------------
%	WORK EXPERIENCE
%----------------------------------------------------------------------------------------

	\begin{rSection}{Experience}

		% \begin{rSubsection}{Troph}{Dec. 2019---Now}{Backend Developer Intern}{Remote}
		% 	\item Being responsible for developing server Backend and populate REST API for the Troph IRC website.
		% \end{rSubsection}

	%------------------------------------------------ 

		\begin{rSubsection}{PLSE Lab, University of Washington}{Oct. 2019---Now}{Research Assistant}{Seattle, WA} 
			% \item Working on building an automated and verified compiler for Deep Learning acclelerators by integrating \textbf{ILA Codegen} (Instruction-level Abstraction) to \textbf{TVM} with Bring Your Own Codegen (BYOC).
			\item Implemented a Just-in-time compiler (\textbf{JIT}) in \href{https://tvm.apache.org/}{\color{blue}\myul{TVM}} that can offload deep learning operators, perform simulation-based verification during compile time and address granularity mismatches between high-level intermediate representations and target accelerators by compiling to \href{https://github.com/PrincetonUniversity/ILAng}{\color{blue}\myul{Instruction-level Abstraction}} (\textbf{ILA}).
			% \item Worked with \textbf{TVM} and proposed bug fixes in \textbf{model quantization} and \textbf{AoT Compiler}.
			% \item Implemented experiment profiler to investigate \textbf{compilation overhead} and solved performance discrepancy by pinning workload on single core complex (CCX).
			\item Worked on \href{https://github.com/uwsampl/dtr-prototype}{\color{blue} \myul{Dynamic Tensor Rematerialization}} (\textbf{DTR}), an greedy gradient checkpointing algorithm that \textbf{enables} training Deep Learning models on memory-constrained devices (e.g. GPUs, FPGA-based accelerators) by evicting and recomputing tensors \textbf{on fly} without prior knowledge about the model architecture. \item Implemented continuous integration and evaluation for the DTR implementation in \textbf{PyTorch}; the infrastructure was able to detect multiple performance regression and expose bugs in operator dispatcher.
		\end{rSubsection}
		% \vspace{-5pt}

		\begin{rSubsection}{Paul G. Allen School, University of Washington}{Mar. 2021---Jun. 2021}{Teaching Assistant}{Seattle, WA}
			\item Worked as TA for \textbf{Principle of Programming Languages} (CSE 505)
			\item Helped re-designing CSE 505 and developing course materials for various topics about PL and formal verification (\textbf{Hoare Logic}, \textbf{Lambda Calculus} and \textbf{System F}, etc.) in \textbf{Coq}.
			\item Held office hours and shared tricks used in Coq tactics and Coq programming.
			\item Coordinate grading of all homework assignments.
		\end{rSubsection}

	%-------------------------------------------------

		% \begin{rSubsection}{ECE, University of Washington}{Jan. 2019---Sept. 2019}{Research Assistant}{Seattle, WA}
		% 	\item Developed an online panel for visualizing data collected from solar panels deployed around UW campus.
		% \end{rSubsection}
		\vspace{-5pt}

	%------------------------------------------------

		% \begin{rSubsection}{HCC Computing Community}{Feb. 2015--Mar. 2018}{President \& Lecturer \& Web Developer}{Bejing, China}
		% 	\item Developed \& maintained a website named Shiyiquan for Beijing National Day School to manage student organizations. Designed courses and lectures for high school students to learn Python, Android development and functional programming and held classes for students interested in computer programming.
		% \end{rSubsection}
		% \vspace{-5pt}
	\end{rSection}
	\vspace{-5pt}
	% \newpage
	%----------------------------------------------------------------------------------------
	%	PROJECTS
	%----------------------------------------------------------------------------------------

	\begin{rSection}{Projects}
	
        %------------------------------
        %           BNDSOJ
        %------------------------------
		% \textbf{BNDS Online Judge} \hfill {\em {\href{https://github.com/AD1024/BNDSOJ}{On Github}}}
		% \vspace{-5pt}

		% An online judge designed for Beijing National Day School, developed based on \href{https://github.com/vfleaking/uoj}{Universal Online Judge (UOJ)}. It is optimized for CS course instructors to assign, review and grade homework for students. Implementations are contributed to UOJ open source community. Currently, it is being used as the official testing website at BNDS.
		% \vspace{-5pt}

		%------------------------------
        %          Realm.js
        %------------------------------
		% \textbf{Realm.js}
		% \vspace{-5pt}

		% A functional reactive programming library designed for frontend developers. Written in TypeScript, the library combines the design of Elm and RxJs. It adopts a functional imperative and reactive hybrid paradigm of frontend programming. It is the final project of CSE 402 (Domain-Specific Language).
		% \vspace{-5pt}

		%------------------------------
		%            Sager
		%------------------------------
		% \textbf{\href{https://github.com/AD1024/Sager}{Sager}} % \hfill {\em {\href{https://github.com/AD1024/Sager}{On GitHub}}}
		% \vspace{-5pt}
		% \begin{itemize}
		% 	\setlength{\itemsep}{1pt}
        %     \setlength{\parskip}{0pt}
		% 	\setlength{\parsep}{0pt}
		% 	\item A demonic data structure synthesizer written in \textsc{rosette}, a \textbf{solver-aided} language based on \textbf{Racket}. \textbf{Sager} is able to exploit algorithm bottleneck by performing \textbf{symbolic exeuction} over the whole algorithm and using \textbf{SMT solver} to synthesize a sample data structure the algorithm works on that pushes the algorithm to its worst-case performance.
			% \item A demonic data structure synthesizer that aims to explore worst-cases performance of graph algorithms.
			% \item \textbf{Language} \& \textsc{Tools}: \textbf{Racket}, \textbf{Rosette}, \textsc{Z3}
			% \item Keywords: SMT Solver, Incremental Solving, Program Synthesis, Symbolic Execution
		% \end{itemize}
		% \vspace{-5pt}

		%------------------------------
		%      veripy
		%------------------------------
		\textbf{\href{https://github.com/AD1024/veripy}{\color{blue} \myul{veripy}}} % \hfill {\em {\href{https://github.com/AD1024/veripy}{On GitHub}}}
		\vspace{-5pt}

		\begin{itemize}
			\setlength{\itemsep}{1pt}
            \setlength{\parskip}{0pt}
			\setlength{\parsep}{0pt}
			\item An easy-to-use auto-active program verification library for Python programs written in \textbf{Python}.
			\item The library is shallowly embedded in Python and the interface is implemented as \textbf{decorators}. It compiles annotated Python functions to \textbf{SMT} formulae and calls \textbf{SMT solver} to check whether it matches the given specification and gives a counter-example input when it violates any constraint.
			% \item \textbf{Language} \& \textsc{Tools}: \textbf{Python 3}, \textbf{SMT-LIB}, \textsc{Z3}, \textsc{pyparsing}
			% \item Keywords: SMT Solver, Static Analysis, Hoare Logic, Program Verification
		\end{itemize}

		%------------------------------
		%      dtlc
		%------------------------------
		\textbf{\href{https://github.com/AD1024/dtlc}{\color{blue} \myul{dtlc}}} % \hfill {\em {\href{https://github.com/AD1024/dtlc}{On GitHub}}}
		\vspace{-5pt}

		\begin{itemize}
			\setlength{\itemsep}{1pt}
            \setlength{\parskip}{0pt}
			\setlength{\parsep}{0pt}
			\item Implemented \textbf{dependently-typed} lambda calculus in Martin-Löf style intuitionistic type theory.
			\item Written in \textbf{OCaml}, \textbf{dtlc} has a language frontend \textbf{Lexer \& Parser} implemented using \textbf{Menhir}. Core language supports \textbf{type unification} with \textbf{metavariables} which makes the type inference stronger.
			\item Implemented eliminators for \textbf{naturals}, \textbf{identity type}, \textbf{union type}, etc.
			% \item \textbf{Language} \& \textsc{Tools}: \textbf{OCaml}, \textsc{Menhir}, \textsc{Dune}
			% \item Keywords: Type Theory - Dependent Type, Proof Assistant, Functional Programming
		\end{itemize}

		%------------------------------
		%            Blog
		%------------------------------
		% \textbf{Blog} \hfill {\em {\href{https://github.com/BNDS-Programmers/Blog}{On Github}}}
		% \vspace{-5pt}

		% A general-purpose blog system built by using Koa2 that supports both RTF and Markdown; renders mathematical expressions using KaTeX; supports code highlight with hightlight.js; has multi-user management functionality; provides customization supports for about page. 

        %------------------------------
        %         Lotus Leaf
        %------------------------------
		% \textbf{Lotus Leaf} \hfill {\em {\href{https://github.com/AD1024/lotus-leaf-frontend}{On Github}}}
		% \vspace{-5pt}

		% A frontend of a data panel for visualizing and monitoring data sets collected from solar metrics deployed arround University of Washington campus; The system is deployed at the Clear Energy Institute at University Village. This is a research project of ECE Department instructed by Professor Kirschen.

	\end{rSection}
	\vspace{-5pt}

	\begin{rSection}{Publications}
		\begin{enumerate}
			\setlength{\itemsep}{1pt}
            \setlength{\parskip}{0pt}
			\setlength{\parsep}{0pt}
			\item \bibentry{kirisame2021dynamic}
			\item \bibentry{latte21}
		\end{enumerate}
	\end{rSection}
	
	%----------------------------------------------------------------------------------------
	%	HONORS
	%----------------------------------------------------------------------------------------
	\vspace{-5pt}

	% \begin{rSection}{Honors}
	% 	\begin{itemize}
	% 		\setlength{\itemsep}{1pt}
	% 		\setlength{\parskip}{0pt}
	% 		\setlength{\parsep}{0pt}
	% 		\item \textbf{Lynn Conway Research Award (DTR Team)}, \href{https://adacenter.org}{ADA Center} \hfill {\em 2020}
	% 		% \item \textbf{JASSO Scholarship}, Waseda University \hfill {\em Jun. 2019}
	% 		\item \textbf{Annual Dean's List}, University of Washington \hfill {\em 2018---2021}
	% 		% \item \textbf{Global Rank 16 (out of 300+)}, High School CTF 5 \hfill {\em May. 2018}
	% 		% \item \textbf{Second Prize}, National Software Development Contest for High School Students \hfill {\em Jul. 2017}
	% 		% \item \textbf{First Prize}, Software Engineering Hackathon for High School Students (Beijing) \hfill {\em Apr. 2017}
	% 		% \item \textbf{Global Rank 33 (out of 400+)}, High School CTF 4 \hfill {\em May. 2017}
	% 		% \item \textbf{Second Prize}, National Olympiad in Informatics (Beijing Regional) \hfill {\em Dec. 2016}
	% 		% \item \textbf{Outstanding Developer Team}, Xiaomi Inc. \hfill {\em Mar. 2015}
	% 	\end{itemize}
	% \end{rSection}
\end{document}
