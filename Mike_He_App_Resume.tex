\documentclass{resume}
\usepackage{hyperref}

\usepackage[left=0.40in,top=0.3in,right=0.40in,bottom=0.3in]{geometry}
\usepackage{titlesec}
\usepackage{enumitem}
\usepackage{natbib, bibentry}
\titlespacing*{\rSection}{0pt}{1.1\baselineskip}{\baselineskip}
% \name{Mike He}
% \address{1135 NE CAMPUS PKWY \\ Seattle, WA 98105}
% \address{(206)~$\cdot$~887~$\cdot$~8588 \\ dh63@cs.washington.edu}
% \address{https://ad1024.space}
\newcommand{\myul}[2][blue]{\setulcolor{#1}\ul{#2}\setulcolor{blue}}
\usepackage{xcolor,soul,lipsum}
\begin{document}
\vspace{-0.5in}
\bibliographystyle{unsrt}
\nobibliography{publication.bib}
	\MakeUppercase{\Large{\textbf{Mike He}}} \hfill {\em{\href{mailto:dh63@cs.washington.edu}{dh63@cs.washington.edu}}}\\
	\vspace{-5pt}\href{https://homes.cs.washington.edu/~dh63/}{https://homes.cs.washington.edu/$\sim$dh63/} \hfill{\em (206)~$\cdot$~887~$\cdot$~8588}


%----------------------------------------------------------------------------------------
%	Specifications
%----------------------------------------------------------------------------------------

	% \begin{rSection}{Skills}
	% 	\begin{tabular}{ @{} >{\bfseries}l @{\hspace{4ex}} l }
	% 		Languages & C/C++, Python, Java, Rust, OCaml, Haskell, Coq, Agda, \LaTeX \\
	% 		Skills    & Certified \& Functional Programming, Automated Verification \\
	% 		Others    & I've been playing the violin for 17 years. I like Symphonies composed by \href{https://imslp.org/wiki/Category:Mahler,_Gustav}{Gustav Mahler}.
	% 	\end{tabular}
	% \end{rSection}
	% \vspace{-5pt}
%----------------------------------------------------------------------------------------
%	EDUCATION
%----------------------------------------------------------------------------------------

	\begin{rSection}{Education}
	{\bf University of Washington, Seattle} \hfill {\em Sept. 2018---Est. Jun. 2022} \\
	\textit{B.S. in Computer Science}
	\vspace{-5pt}
        \begin{itemize}[leftmargin=*]
            \setlength{\itemsep}{1pt}
            \setlength{\parskip}{0pt}
			\setlength{\parsep}{0pt}
			\item Major GPA: 3.78 / 4.00
            \item Fields of Study: Programming Languages \& Formal Verification \& Compilers \& MLSys
            \item Honors: Dean's List (College of Art \& Science) 2018--Now; CRA Outstanding Undergraduate Researcher Award nominee of the Allen School (2021)
		\end{itemize}
	\end{rSection}
	\vspace{-5pt}
    \begin{rSection}{Publications}
		\begin{enumerate}
			\setlength{\itemsep}{1pt}
            \setlength{\parskip}{0pt}
			\setlength{\parsep}{0pt}
			\item \bibentry{kirisame2021dynamic} (\small{*: Equal Contribution})
			\item \textbf{(Under review at PLDI '22)} \bibentry{pldi22}
		\end{enumerate}
        \vspace{-5pt}
	\end{rSection}
	\vspace{1pt}
	\begin{rSection}{Workshop Papers}
		\begin{enumerate}
			\setlength{\itemsep}{1pt}
            \setlength{\parskip}{0pt}
			\setlength{\parsep}{0pt}
			\item \bibentry{latte21} (\small{*: Equal Contribution})
		\end{enumerate}
        \vspace{-5pt}
	\end{rSection}
    \vspace{1pt}

    \begin{rSection}{Experience}
        \begin{rSubsection}{3LA, LATTE '21}{June. 2020---Now}{Research Assistant @ PLSE}{Seattle, WA}
            \item \href{https://capra.cs.cornell.edu/latte21/paper/30.pdf}{\color{blue} \myul{3LA}} proposes an end-to-end compilation flow that provides \textbf{flexible} and \textbf{verifiable} compiler support for custom Deep Learning (\textbf{DL}) accelerators. 3LA has a builtin implementation agnostic pattern matching algorithm that is capable of find accelerator supported workloads in DL models leveraging the power of Equality Saturation. Moreover, 3LA addresses the mapping gap between DL models represented in high-level domain-specific languages (DSLs) and specialized accelerators using Instruction-level Abstraction (\textbf{ILA}) as the software-hardware interface.
            \item \textbf{Talks \& Presentations}:
            \vspace{-5pt}
                \begin{enumerate}
                    \setlength{\itemsep}{1pt}
                    \setlength{\parskip}{0pt}
                    \setlength{\parsep}{0pt}
                    \item \textit{From DSLs to Accelerator-rich Platform: Addressing the Mapping Gap}, Sept. 2021 at Intel (presented jointly with \href{https://homes.cs.washington.edu/~sslyu/}{\color{blue} \myul {Steven Lyubomirsky}})
                    \item \textit{Correct \& Flexible Compiler Support for Custom Accelerators}, Sept. 2021 at SRC ADA Center
                \end{enumerate} 
        \end{rSubsection}
        \vspace{-5pt}
        \begin{rSubsection}{Dynamic Tensor Rematerialization, ICLR '21}{Oct. 2019---Aug. 2021}{Research Assistant @ PLSE}{Seattle, WA}
            \item \href{https://github.com/uwsampl/dtr-prototype}{\color{blue} \myul{Dynamic Tensor Rematerialization}} (\textbf{DTR}) is a greedy gradient checkpointing algorithm. DTR \textbf{enables} training Deep Learning models on memory-constrained devices. Unlike previous approaches, DTR does not need any information of the DL model architectures ahead-of-time; instead it saves memory by evicting and recomputing tensors \textbf{on the fly}, i.e. trading time for memory, which further exploit opportunities of using gradient checkpointing on DL trainings. DTR is comparably efficient as previous approaches: it requires only $\mathcal{O}(N)$ more forward computations when training a $N$-layer linear feed-forward neural network with an $\Omega(\sqrt{N})$ memory budget.
        \end{rSubsection}
        \vspace{-5pt}
		\begin{rSubsection}{Paul G. Allen School, University of Washington}{Mar. 2021---June. 2021}{Teaching Assistant}{Seattle, WA}
			\item Worked as a TA for \textbf{Principles of Programming Languages} (CSE 505)
			\item Helped re-designing CSE 505 and developing course materials for various topics about PL and formal verification (\textbf{Hoare Logic}, \textbf{Lambda Calculus} and \textbf{System F}, etc.) in \textbf{Coq}.
		\end{rSubsection}
    \end{rSection}
	\vspace{-5pt}
	\begin{rSection}{Service}
    	\begin{enumerate}
    		\setlength{\itemsep}{1pt}
            \setlength{\parskip}{0pt}
    		\setlength{\parsep}{0pt}
    		\item[$\rightarrow$] \textbf{MICRO '21}, Artifact Evaluation
    	\end{enumerate}
	\end{rSection}
\end{document}
