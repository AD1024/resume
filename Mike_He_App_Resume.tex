\documentclass{resume}
\usepackage{hyperref}

\usepackage[left=0.75in,top=0.65in,right=0.75in,bottom=0.65in]{geometry}
\usepackage{titlesec}
\usepackage{enumitem}
\usepackage{natbib, bibentry}
\titlespacing*{\rSection}{0pt}{1.1\baselineskip}{\baselineskip}
% \name{Mike He}
% \address{1135 NE CAMPUS PKWY \\ Seattle, WA 98105}
% \address{(206)~$\cdot$~887~$\cdot$~8588 \\ dh63@cs.washington.edu}
% \address{https://ad1024.space}
\newcommand{\myul}[2][blue]{\setulcolor{#1}\ul{#2}\setulcolor{blue}}
\usepackage{xcolor,soul,lipsum}
\begin{document}
\vspace{-0.5in}
\bibliographystyle{unsrt}
\nobibliography{publication.bib}
	\MakeUppercase{\Large{\textbf{Deyuan (Mike) He}}} \hfill {\em{\href{mailto:mikehe@princeton.edu}{mikehe@princeton.edu}}}\\
	\vspace{-5pt}\href{https://homes.cs.washington.edu/~dh63/}{https://homes.cs.washington.edu/$\sim$dh63/} \hfill{\em (206)~$\cdot$~887~$\cdot$~8588}
%----------------------------------------------------------------------------------------
%	EDUCATION
%----------------------------------------------------------------------------------------

	\begin{rSection}{Education}
    {\bf Princeton University, Princeton, NJ}\hfill {\em Sept. 2022---Est 2027} \\
    \textit{Ph.D. in Computer Science}
    \vspace{-5pt}
    \begin{itemize}[leftmargin=*]
        \setlength{\itemsep}{1pt}
        \setlength{\parskip}{0pt}
        \setlength{\parsep}{0pt}
        \item Advisor(s): TBD
        \item Fields of Study: Programming Languages \& Formal Verification \& Compilers \& MLSys
    \end{itemize}
	{\bf University of Washington, Seattle, WA} \hfill {\em Sept. 2018---Jun. 2022} \\
	\textit{B.S. in Computer Science}
	\vspace{-5pt}
        \begin{itemize}[leftmargin=*]
            \setlength{\itemsep}{1pt}
            \setlength{\parskip}{0pt}
			\setlength{\parsep}{0pt}
			\item GPA: 3.89 (ranking not applicable)
            \item Fields of Study: Programming Languages \& Formal Verification \& Compilers \& MLSys
            \item Honors: CRA Outstanding Undergraduate Researcher Award 2022 (Honorable Mention)
		\end{itemize}
	\end{rSection}
	\vspace{-5pt}
    \begin{rSection}{Publications}
		\begin{enumerate}
			\setlength{\itemsep}{1pt}
            \setlength{\parskip}{0pt}
			\setlength{\parsep}{0pt}
			\item \bibentry{kirisame2021dynamic} (\small{*: Equal Contribution})
			\item \bibentry{pldi22}
		\end{enumerate}
        \vspace{-5pt}
	\end{rSection}
	\vspace{1pt}
	\begin{rSection}{Workshop Papers}
		\begin{enumerate}
			\setlength{\itemsep}{1pt}
            \setlength{\parskip}{0pt}
			\setlength{\parsep}{0pt}
			\item \bibentry{latte21} (\small{*: Equal Contribution})
		\end{enumerate}
        \vspace{-5pt}
	\end{rSection}
    \vspace{1pt}

    \begin{rSection}{Experience}
		\begin{rSubsection}{Taichi Graphics}{Jun. 2022---Sept. 2022}{Compiler R\&D Intern}{Remote}
			\item Working on optimizations for high-level intermediate representation of Taichi.
			\item Exploring equality saturation for Taichi.
		\end{rSubsection}
		\begin{rSubsection}{Intel Labs}{Mar. 2022---Jun. 2022}{Formal Verification Research Intern}{Hillsboro, OR \& Remote}
			\item Implemented Pyrope, an embedded domain-specific language that aims to enable correct-by-construction hardware modeling. Pyrope captures a major subset of Python syntax and, in addition, provides interfaces for proof-carrying programming in Python. Pyrope compiles to Dafny for automated verification.
            \item Encoded the (multi-)montgomery reduction algorithm and successfully verified by compiling to Dafny.
		\end{rSubsection}
		\vspace{-5pt}
        \begin{rSubsection}{3LA, LATTE '21}{Jun. 2020---Now}{Research Assistant @ PLSE}{Seattle, WA}
            \item \href{https://capra.cs.cornell.edu/latte21/paper/30.pdf}{\color{blue} \myul{3LA}} proposes an end-to-end compilation flow that provides \textbf{flexible} and \textbf{verifiable} compiler support for custom Deep Learning (\textbf{DL}) accelerators.
            \item Implemented Flexible Matching: using equality saturation to search for optimal operator offloading to accelerators.
        \end{rSubsection}
        \vspace{-5pt}
        \begin{rSubsection}{Dynamic Tensor Rematerialization, ICLR '21}{Oct. 2019---Aug. 2021}{Research Assistant @ PLSE}{Seattle, WA}
            \item \href{https://github.com/uwsampl/dtr-prototype}{\color{blue} \myul{Dynamic Tensor Rematerialization}} (\textbf{DTR}) is a greedy gradient checkpointing algorithm. DTR \textbf{enables} training Deep Learning models on memory-constrained devices.
            \item Implemented evaluations and nightly CI for DTR prototype in PyTorch.
            \item Prepared submission artifact of DTR to ICLR '21.
        \end{rSubsection}
        % \vspace{-5pt}
		% \begin{rSubsection}{Paul G. Allen School, University of Washington}{Mar. 2021---June. 2021}{Teaching Assistant}{Seattle, WA}
		% 	\item Worked as a TA for \textbf{Principles of Programming Languages} (CSE 505)
		% 	\item Helped re-designing CSE 505 and developing course materials for various topics about PL and formal verification (\textbf{Hoare Logic}, \textbf{Lambda Calculus} and \textbf{System F}, etc.) in \textbf{Coq}.
		% \end{rSubsection}
    \end{rSection}
	\vspace{-5pt}
    \begin{rSection}{Teaching}
        \begin{rSubsection}{Paul G. Allen School, University of Washington}{Mar. 2021---Jun. 2021}{Teaching Assistant}{Seattle, WA}
			\item Worked as a TA for \textbf{Principles of Programming Languages} (CSE 505)
			\item Helped re-designing CSE 505 and developing course materials for various topics about PL and formal verification (\textbf{Hoare Logic}, \textbf{Lambda Calculus} and \textbf{System F}, etc.) in \textbf{Coq}.
		\end{rSubsection}
    \end{rSection}
    \vspace{-5pt}
	\begin{rSection}{Talks \& Presentations}
		\vspace{-5pt}
		\begin{enumerate}
			\setlength{\itemsep}{1pt}
			\setlength{\parskip}{0pt}
			\setlength{\parsep}{0pt}
            \item \textit{Pyrope: Towards Provably Correct Hardware Modeling in Python/HeteroCL}, Jun. 2nd at Intel.
			\item \textit{From DSLs to Accelerator-rich Platform: Addressing the Mapping Gap}, Sept. 2021 at Intel (presented jointly with \href{https://homes.cs.washington.edu/~sslyu/}{\color{blue} \myul {Dr. Steven Lyubomirsky}})
			\item \textit{Correct \& Flexible Compiler Support for Custom Accelerators}, Sept. 2021 at SRC ADA Center
		\end{enumerate} 
	\end{rSection}
	\vspace{-5pt}
	\begin{rSection}{Conference Service}
    	\begin{enumerate}
    		\setlength{\itemsep}{1pt}
            \setlength{\parskip}{0pt}
    		\setlength{\parsep}{0pt}
    		\item[$\rightarrow$] \textbf{MICRO '21}, Artifact Evaluation
    	\end{enumerate}
	\end{rSection}
    \begin{rSection}{Projects}
        %------------------------------
		%            Paxos
		%------------------------------
        % \textbf{Paxos on KVStore Servers}
        % \vspace{-5pt}
        % \begin{itemize}
		% 	\setlength{\itemsep}{1pt}
        %     \setlength{\parskip}{0pt}
		% 	\setlength{\parsep}{0pt}
		% 	\item Implemented the Paxos consensus algorithm on KVStore servers
		% 	\item Course project of \textbf{CSE 452 - Distributed System} (UW)
		% 	\item \textbf{Language} \& \textsc{Tools}: \textbf{Java}
		% 	\item Keywords: Network Systems, Distributed Systems
		% \end{itemize}
		% \vspace{-5pt}
        %------------------------------
		%            Sager
		%------------------------------
		\textbf{\href{https://github.com/AD1024/Sager}{\color{blue} \myul{Sager}}} % \hfill {\em {\href{https://github.com/AD1024/Sager}{On GitHub}}}
		\vspace{-5pt}
		\begin{itemize}
			\setlength{\itemsep}{1pt}
            \setlength{\parskip}{0pt}
			\setlength{\parsep}{0pt}
			\item A demonic data structure synthesizer written in \textsc{rosette}, a \textbf{solver-aided} language based on \textbf{Racket}. \textbf{Sager} is able to exploit algorithm bottleneck by performing \textbf{symbolic exeuction} over the whole algorithm and using \textbf{SMT solver} to synthesize a sample data structure the algorithm works on that pushes the algorithm to its worst-case performance.
			\item A demonic data structure synthesizer that aims to explore worst-cases performance of graph algorithms.
			\item \textbf{Language} \& \textsc{Tools}: \textbf{Racket}, \textbf{Rosette}, \textsc{Z3}
			\item Keywords: SMT Solver, Incremental Solving, Program Synthesis, Symbolic Execution
		\end{itemize}
		\vspace{-5pt}

		%------------------------------
		%      veripy
		%------------------------------
		\textbf{\href{https://github.com/AD1024/veripy}{\color{blue} \myul{veripy}}} % \hfill {\em {\href{https://github.com/AD1024/veripy}{On GitHub}}}
		\vspace{-5pt}

		\begin{itemize}
			\setlength{\itemsep}{1pt}
            \setlength{\parskip}{0pt}
			\setlength{\parsep}{0pt}
			\item An easy-to-use auto-active program verification library for Python programs written in \textbf{Python}.
			\item The library is shallowly embedded in Python and the interface is implemented as \textbf{decorators}. It compiles annotated Python functions to \textbf{SMT} formulae and calls \textbf{SMT solver} to check whether it matches the given specification and gives a counter-example input when it violates any constraint.
			\item \textbf{Language} \& \textsc{Tools}: \textbf{Python 3}, \textbf{SMT-LIB}, \textsc{Z3}, \textsc{pyparsing}
			\item Keywords: SMT Solver, Static Analysis, Hoare Logic, Program Verification
		\end{itemize}

		%------------------------------
		%      dtlc
		%------------------------------
		\textbf{\href{https://github.com/AD1024/dtlc}{\color{blue} \myul{dtlc}}} % \hfill {\em {\href{https://github.com/AD1024/dtlc}{On GitHub}}}
		\vspace{-5pt}

		\begin{itemize}
			\setlength{\itemsep}{1pt}
            \setlength{\parskip}{0pt}
			\setlength{\parsep}{0pt}
			\item Implemented \textbf{dependently-typed} lambda calculus in Martin-Löf style intuitionistic type theory.
			\item Written in \textbf{OCaml}, \textbf{dtlc} has a language frontend \textbf{Lexer \& Parser} implemented using \textbf{Menhir}. Core language supports \textbf{type unification} with \textbf{metavariables} which makes the type inference stronger.
			\item Implemented eliminators for \textbf{naturals}, \textbf{identity type}, \textbf{union type}, etc.
			\item \textbf{Language} \& \textsc{Tools}: \textbf{OCaml}, \textsc{Menhir}, \textsc{Dune}
			\item Keywords: Type Theory - Dependent Type, Proof Assistant, Functional Programming
		\end{itemize}
    \end{rSection}
\end{document}
